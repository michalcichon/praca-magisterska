\chapter{Podsumowanie i zakończenie}

Różnorodność technik służących zarówno do ekstrakcji cech obrazów jak i klasyfikacji oraz dostępność dobrze udokumentowanych i zarazem darmowych bibliotek do przetwarzania obrazów pozwala na budowanie coraz skuteczniejszych narzędzi służących do kategoryzacji obrazów. Narzędzia te mogą być bardzo przydatne nie tylko do zapanowania nad zbiorami zdjęć, ale także mogą się przydać do budowy systemów zaawansowanego wyszukiwania grafiki na podstawie zapytań lub do szczegółowej analizy przeglądanych zdjęć w Internecie.

Narzędzie do kategoryzacji zdjęć zaproponowane w niniejszej pracy zostało oparte na metodzie \emph{Bag of Visual Words} i daje bardzo dobre wyniki dla 2 lub 4 kategorii. Wyniki przeprowadzonych testów pozwalają mieć nadzieję na to, że skuteczność metody dla przypadku większej ilości kategorii można poprawić poprzez dostrojenie parametrów maszyny wektorów nośnych oraz powiększenie zbiorów treningowych.



\section{Możliwe dalsze kierunki rozwoju}

