\chapter{Podsumowanie i zakończenie}

Różnorodność technik służących zarówno do ekstrakcji cech obrazów jak i klasyfikacji oraz dostępność dobrze udokumentowanych i zarazem darmowych bibliotek do przetwarzania obrazów pozwala na budowanie coraz skuteczniejszych narzędzi do kategoryzacji obrazów. Narzędzia te mogą być bardzo przydatne nie tylko do zapanowania nad rozrastającymi się zbiorami zdjęć, ale także mogą się przydać do budowy systemów zaawansowanego wyszukiwania grafiki na podstawie zapytań lub do szczegółowej analizy przeglądanych zdjęć w Internecie.

Narzędzie do kategoryzacji zdjęć zaproponowane w niniejszej pracy zostało oparte na metodzie \emph{Bag of Visual Words} i daje bardzo dobre wyniki dla 2 lub 4 kategorii. Wyniki przeprowadzonych testów pozwalają mieć nadzieję na to, że skuteczność metody dla przypadku większej ilości kategorii można poprawić poprzez dostrojenie parametrów maszyny wektorów nośnych oraz powiększenie zbiorów treningowych.

Zaproponowane API jest bardzo proste i umożliwia wykonanie podstawowych operacji na modelu oraz kategoryzację dowolnego obrazu. Silnik umożliwia też testowanie jakości kategoryzacji poprzez wskazanie trafności dla przykładowego scenariusza testowego. 

\section{Możliwe dalsze kierunki rozwoju}

Aby podnieść skuteczność zaproponowanego rozwiązania można by było zastosować kilka dodatkowych technik.

Zastosowanie dodatkowych deskryptorów, które są nieskorelowane z SIFT lub SURF mogłoby podnieść jakość kategoryzacji. Możliwe byłoby również zastosowanie deskryptorów GIST, które niejako z definicji nie służą zastąpieniu SIFT/SURF ale mogą równorzędnie wesprzeć proces kategoryzacji. Prawdopodobnie również zastosowanie LBP pozwoliłoby na polepszenie wyników.  

Dodatkowo wyznaczanie parametrów maszyny wektorów nośnych za pomocą prób i błędów można by zastąpić bardziej wyrafinowanymi sposobem opartym na walidacji krzyżowej.