% Praca magisterska - Michał Cichoń, AGH 2014
% Title: Silnik do automatycznej kategoryzacji obrazów
\documentclass[a4paper,12pt]{book}
\usepackage[utf8]{inputenc}
\usepackage[T1]{polski}
\usepackage{polski}
\usepackage{color}
\usepackage{helvet}
\usepackage{graphicx}
\usepackage{geometry}
\usepackage{epstopdf}
\usepackage{titlesec}
\usepackage{indentfirst}
\usepackage{verbatim}
\usepackage{moreverb}
\usepackage{nameref}
\usepackage{listings}
\usepackage{url}
\usepackage[font=footnotesize,labelfont=bf]{caption}
\geometry{hmargin={2cm, 2cm}, height=10.0in}
\assignpagestyle{\chapter}{empty}
\linespread{1.3} %interlinia ustawiona na 1.5; 1.3 (LaTeX) to 1.5, 1.6 (LaTeX) to 2

\newenvironment{dedication}
{
   \cleardoublepage
   \thispagestyle{empty}
   \vspace*{\stretch{10}}
   \hfill\begin{minipage}[t]{0.66\textwidth}
   \raggedright
}%
{
   \end{minipage}
   \vspace*{\stretch{3}}
   \clearpage
}



\begin{document}
\nocite{*}

% =====  STRONA TYTULOWA PRACY MAGISTERSKIEJKIEJ ====
% ostatnia modyfikacja: 2009/07/01, K. Malarz

\thispagestyle{empty}
%% ------------------------ NAGLOWEK STRONY ---------------------------------
\includegraphics[height=37.5mm]{agh_nzw_a_pl_1w_wbr_cmyk.eps}\\
\rule{30mm}{0pt}
{\large \textsf{Wydział Fizyki i Informatyki Stosowanej}}\\
\rule{\textwidth}{3pt}\\
\rule[2ex]
{\textwidth}{1pt}\\
\vspace{7ex}
\begin{center}
{\LARGE \bf \textsf{Praca magisterska}}\\
\vspace{13ex}
% --------------------------- IMIE I NAZWISKO -------------------------------
{\bf \Large \textsf{Michał Cichoń}}\\
\vspace{3ex}
{\sf\small kierunek studiów:} {\bf\small \textsf{informatyka stosowana}}\\
\vspace{1.5ex}
{\sf\small specjalność:} {\bf\small \textsf{grafika komputerowa i przetwarzanie obrazów}}\\
\vspace{10ex}
%% ------------------------ TYTUL PRACY --------------------------------------
{\bf \huge \textsf{Silnik do automatycznej kategoryzacji obrazów}}\\
\vspace{14ex}
%% ------------------------ OPIEKUN PRACY ------------------------------------
{\Large Opiekun: \bf \textsf{dr inż.\ Maciej Śniechowski}}\\
\vspace{22ex}
{\large \bf \textsf{Kraków, wrzesień 2014}}
\end{center}
%% =====  STRONA TYTUŁOWA PRACY MAGISTERSKIEJKIEJ ====

\newpage

%% =====  TYŁ STRONY TYTUŁOWEJ PRACY MAGISTERSKIEJKIEJ ====
{\sf Oświadczam, świadomy odpowiedzialności karnej za poświadczenie nieprawdy, że niniejszą pracę dyplomową wykonałem osobiście i samodzielnie i  nie korzystałem ze źródeł innych niż wymienione w pracy.}

\vspace{14ex}

\begin{center}
\begin{tabular}{lr}
~~~~~~~~~~~~~~~~~~~~~~~~~~~~~~~~~~~~~~~~~~~~~~~~~~~~~~~~~~~~~~~~~ &
................................................................. \\
~ & {\sf (czytelny podpis)}\\
\end{tabular}
\end{center}

%% =====  TYL STRONY TYTULOWEJ PRACY MAGISTERSKIEJKIEJ ====

\newpage
\rightline{Kraków, ?? czerwca 2014}
\begin{center}
{\bf Tematyka pracy magisterskiej i praktyki dyplomowej
Michała Cichonia,
studenta V roku studiów kierunku informatyka stosowana, specjalności grafika komputerowa i przetwarzanie obrazów.}\\
\end{center}

Temat pracy magisterskiej:
{\bf Silnik do automatycznej kategoryzacji obrazów}\\

\begin{tabular}{rl}

Opiekun pracy:                  & dr inż.\ Maciej Śniechowski\\
Recenzenci pracy:               & ...\\
Miejsce praktyki dyplomowej:    & WFiIS AGH, Kraków\\
\end{tabular}

\begin{center}
{\bf Program pracy magisterskiej i praktyki dyplomowej}
\end{center}

\begin{enumerate}
\item Omówienie realizacji pracy magisterskiej z opiekunem.
\item Zebranie i opracowanie literatury dotyczącej tematu pracy.
\item Praktyka dyplomowa:
\begin{itemize}
\item zapoznanie się z ideą...,
\item uczestnictwo w eksperymentach/przygotowanie oprogramowania...,
\item dyskusja i analiza wyników...
\item sporządzenie sprawozdania z praktyki.
\end{itemize}
\item Kontynuacja obliczeń związanych z tematem pracy magisterskiej.
\item Zebranie i opracowanie wyników obliczeń.
\item Analiza wyników obliczeń numerycznych, ich omówienie i zatwierdzenie przez opiekuna.
\item Opracowanie redakcyjne pracy.
\end{enumerate}


\noindent
Termin oddania w dziekanacie: ?? września 2014\\[1cm]

\begin{center}
\begin{tabular}{lcr}
.............................................................. & ~~~ &
.............................................................. \\
(podpis kierownika katedry) & & (podpis opiekuna) \\
\end{tabular}
\end{center}

\newpage

\noindent
Na kolejnych dwóch stronach proszę dołączyć kolejno recenzje pracy popełnione przez Opiekuna oraz Recenzenta (wydrukowane z systemu MISIO i podpisane przez odpowiednio Opiekuna i Recenzenta pracy). Papierową wersję pracy (zawierającą podpisane recenzje) proszę złożyć w dziekanacie celem rejestracji co najmniej na tydzień przed planowaną obroną.

\newpage

\noindent
Na kolejnych dwóch stronach proszę dołączyć kolejno recenzje pracy popełnione przez Opiekuna oraz Recenzenta (wydrukowane z systemu MISIO i podpisane przez odpowiednio Opiekuna i Recenzenta pracy). Papierową wersję pracy (zawierającą podpisane recenzje) proszę złożyć w dziekanacie celem rejestracji co najmniej na tydzień przed planowaną obroną.

\begin{dedication}
Chciałbym złożyć serdeczne podziękowania Panu dr.~Maciejowi Śniechowskiemu za okazaną pomoc, przy realizacji tej pracy magisterskiej. A także moim rodzicom...
\end{dedication}

\tableofcontents

\chapter*{Wstęp}
\addcontentsline{toc}{chapter}{Wstęp}
Zajawka...

\section*{Cel pracy}
\addcontentsline{toc}{section}{Cel pracy}
Lorem ipsum

\section*{Przykłady implementacji}
\addcontentsline{toc}{section}{Przykłady implementacji}
gdfgdgsd

\addcontentsline{toc}{chapter}{Bibliografia}
%\bibliography{praca_magisterska}

%%%%%%%%%%% Bibliografia %%%%%%%%%%%%%%%%%%%%%%%%%% 
\begin{thebibliography}{100} % 100 is a random guess of the total number of 
%references 
 
\bibitem{Boney96} Boney, L., Tewfik, A.H., and Hamdy, K.N., ``Digital 
Watermarks for Audio Signals," \emph{Proceedings of the Third IEEE 
International Conference on Multimedia}, pp. 473-480, June 1996. 
 
\bibitem{MG} Goossens, M., Mittelbach, F., Samarin, \emph{A LaTeX 
Companion}, Addison-Wesley, Reading, MA, 1994. 
 
\bibitem{HK} Kopka, H., Daly P.W., \emph{A Guide to LaTeX}, 
Addison-Wesley, Reading, MA, 1999. 
 
\bibitem{Pan} Pan, D., ``A Tutorial on MPEG/Audio Compression," \emph{IEEE 
Multimedia}, Vol.2, pp.60-74, Summer 1998. 
 
\end{thebibliography} 
%%%%%%%%%%%%% end %%%%%%%%%%%%%%%%%%%%%%%%%%%%%%% 


\end{document}

