\chapter*{Wstęp}

\addcontentsline{toc}{chapter}{Wstęp}
Jednym z największych wyzwań, przed którymi stoi współczesna informatyka, jest przetwarzanie dużych ilości danych. Coraz częściej dane te nie reprezentują informacji liczbowych lub tekstowych, ale zawierają pewien rodzaj obrazu. 

Dobrą ilustracją tego zjawiska jest przykład przeglądarki internetowej Google, gdzie początkowo kładziony był nacisk na wyszukiwanie danych tekstowych. Jednak obecnie jednym z najbardziej dynamicznie rozwijanych zasobów są obrazy. Indeksowanie tego zasobu w latach 2001-2010 zwiększyło się z 250 mln do ponad 10 mld i systematycznie rośnie.\cite{Google2010}

Obrazy cyfrowe mają szereg zastosowań, od bardzo prozaicznych takich jak reprezentacja zwykłych zdjęć wykonanych cyfrowym aparatem fotograficznym, do bardzo zaawansowanych takich jak chociażby diagnostyka medyczna, gdzie informacja zawarta na zdjęciu jest kluczowa i niekiedy decyduje o ludzkim życiu.

Zarządzając dużym zbiorem obrazów często posługujemy się opisem słownym. Opis taki, w~formie nazwy pliku, metadanych EXIF\footnote{EXIF - Exchangeable Image File Format; najpopularniejszy standard metadanych dla plików ze zdjęciami cyfrowymi} lub rekordu w bazie danych, daje nam pewne wyobrażenie o tym co znajduje się na obrazie. Wyobrażenie to może być jednak mylne, lub niedostateczne, by pozwalało nam na poprawne jego zidentyfikowanie.

Pomoc w rozwiązaniu tego problemu daje nam teoria rozpoznawania obrazów, stosunkowo nowa, ale bardzo szybko rozwijająca się gałąź informatyki, która daje nam narzędzia umożliwiające takie przetwarzanie obrazów, by możliwe stało się rozpoznanie przynależności obrazu do pewnej klasy.\cite{Tad91}

Stworzenie systemu informatycznego, który pomógłby skategoryzować obrazy na podstawie ich zawartości mogłoby okazać się bardzo pomocne w wielu dziedzinach, m.in.:
\begin{compactitem}
	\item w diagnostyce medycznej, m.in. do automatycznego diagnozowania różnych chorób\cite{GMM10}\cite{MONTEJO13}\cite{WARWICK10},
	\item w produkcji przemysłowej, do wykrywania defektów\cite{GLAZER08}\cite{SUVDAA2012},
	\item do budowy systemów CBIR (QBIC) służących do zwracania obrazów na podstawie zapytań\cite{LZLM06},
	\item w marketingu, m.in. do tworzenia grup docelowych na podstawie rodzaju oglądanych zdjęć w Internecie\cite{MARKETING12}.
\end{compactitem}

Techniki rozpoznania i klasyfikacji mogą znaleźć zastosowanie wszędzie tam, gdzie spodziewamy się uzyskać z dużej ilości obrazów dodatkowe informacje, które dla człowieka nie są widoczne na pierwszy rzut oka lub tam, gdzie mamy podzielić duży zbiór obrazów na podzbiory odpowiadające pewnym ustalonym przez nas cechom.

\section*{Cel pracy}
\addcontentsline{toc}{section}{Cel pracy}
Celem niniejszej pracy jest stworzenie silnika kategoryzacyjnego. Proponowany system miałby za zadanie przypisać dostarczony do niego obraz do odpowiedniej kategorii. W założeniu kategorie powinny zostać zdefiniowane przed uruchomieniem silnika kategoryzacyjnego, a nie tworzone dynamicznie w trakcie jego działania. System nie ma na celu tworzenie nowych kategorii, ale możliwe najdokładniejsze przypisanie nowych obrazów do kategorii zrozumiałych przez człowieka.

W tym miejscu należy uściślić, że pisząc "obraz" mamy tutaj na myśli obraz graficzny, lub też bardziej precyzyjnie, grafikę rastrową. Zastosowanie opisywanych tutaj technik do obrazów innego rodzaju wychodzi poza zakres pracy.

\section*{Przykłady implementacji}
\addcontentsline{toc}{section}{Przykłady implementacji}
W momencie pisania tej pracy istniało kilka rozwiązań, które można by umieścić w dziedzinie kategoryzacji zdjęć.

Jednym z nich jest PiXiT, czyli oprogramowanie służące do klasyfikacji obrazów napisane w języku Java. Rozwiązanie to wykorzystuje oryginalną metodę opartą na ekstrakcji losowych podokien \emph{(ang. random subwindow extraction)} oraz drzewa decyzyjne. Technika ta została zaproponowana przez Raphaëla Marée jako część jego rozprawy doktorskiej. Oprogramowanie to jest dostępne nieodpłatnie dla zastosowań niekomercyjnych, wymaga jednak bezpośredniego kontaktu z autorem i~podania uzasadnienia użycia\cite{PIXIT}.

Innym przykładem podobnej implementacji jest komercyjny produkt Imagga Technologies z Sofii (Bułgaria). Firma ta udostępnia webowe API pozwalające na przypisanie predefiniowanych kategorii do przesłanego zdjęcia. Dostęp do systemu możliwy jest poprzez stronę WWW lub poprzez REST-owe API\cite{IMAGGA}.

Zastosowanie pewnych technik kategoryzacji obrazów widoczny jest również w najnowszej wersji przeglądarki Google. W 2010 roku pojawiło się narzędzie pozwalające na wyszukiwanie obiektów o określonych kolorach\cite{Google2010}.
