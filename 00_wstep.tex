\chapter*{Wstęp}

%TODO TEN WSTEP WYMAGA POPRAWEK
\addcontentsline{toc}{chapter}{Wstęp}
Jednym z największych wyzwań przed którymi stoi współczesna informatyka jest przetwarzanie dużych ilości danych. Coraz częściej dane te nie reprezentują informacji liczbowych lub tekstowych, ale zawierają pewien rodzaj obrazu. 

Obrazy cyfrowe mają szereg zastosowań, od bardzo prozaicznych takich jak reprezentacja amatorskich zdjęć wykonanych cyfrowym aparatem fotograficznym, do bardzo zaawansowanych takich jak chociażby diagnostyka medyczna, gdzie informacja zawarta na zdjęciu jest kluczowa i niekiedy decyduje o ludzkim życiu.

Zarządzając dużym zbiorem obrazów często posługujemy się opisem słownym. Opis taki, w~formie nazwy pliku lub rekordu w bazie danych, daje nam pewne wyobrażenie o tym co znajduje się na obrazie. Wyobrażenie to może być jednak mylne, lub niedostateczne, by pozwalało nam na poprawne jego zidentyfikowanie.

Pomoc w rozwiązaniu tego problemu daje nam teoria rozpoznawania obrazów, stosunkowo nowa, ale bardzo szybko rozwijająca się gałąź informatyki. 

\section*{Cel pracy}
\addcontentsline{toc}{section}{Cel pracy}
Celem pracy jest stworzenie silnika do kategoryzacji obrazów.

\section*{Przykłady implementacji}
\addcontentsline{toc}{section}{Przykłady implementacji}
gdfgdgsd