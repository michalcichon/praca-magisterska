\chapter{Implementacja}

Silnik kategoryzacyjny został zaimplementowany w języku C++, z użyciem środowiska Visual Studio w wersji 2010 Express, która jest czasowo ograniczona i bezpłatna do zastosowań niekomercyjnych.

\section{Język C++ i Visual Studio}

Język C++ jest językiem programowania ogólnego przeznaczenia. Jest językiem wieloparadygmatowym, czyli umożliwia stosowanie różnych paradygmatów programowania: proceduralnego, obiektowego i generycznego. Został zaprojektowany przez Bjarne Stroustrupa jako rozszerzenie języka C i użyty po raz pierwszy w 1979 roku. Charakteryzuje się obiektowymi mechanizmami abstrakcji danych oraz silną statyczną kontrolą typów.

Początkowo do realizacji zadania miał zostać wykorzystany język Python, ze względu na bogactwo bibliotek do uczenia maszynowego oraz przetwarzania obrazów. Warto wspomnieć szczególnie o dwóch bibliotekach: scikit-learn oraz scikit-image, które pozwalają na rozwiązanie wielu problemów związanych z klasyfikacją obrazów, klastrowaniem lub ekstrakcją cech. Zdecydowano o użyciu języka C++ z dwóch powodów:

\begin{compactitem}
	\item \emph{szybkość działania} -- kod jest kompilowany do kodu maszynowego, natomiast Python jest językiem interpretowanym i przez to czas potrzebny na osiągnięcie podobnych wyników jest większy,
	\item \emph{silne typowanie} -- w przeciwieństwie do Pythona, język C++ jest silnie typowany, co w ocenie autora niniejszej pracy ma wpływ na zachowanie porządku w kodzie źródłowym.
\end{compactitem}

Microsoft Visual Studio jest zintegrowanym środowiskiem programistycznym rozwijanym przez firmę Microsoft. W skład pakietu wchodzą kompilatory kilku języków: C\#, J\#, Visual Basic, F\#, C++ oraz zintegrowany debugger, który działa na poziomie kodu źródłowego, jak i maszyny. Pierwsze wydanie pakietu pojawiło się w 1995 i obecnie najnowszą wersją jest wersja 12 z 2013 roku. 

Kompilator Visual C++ wchodzący w skład pakietu jest specjalną wersją kompilatora języka C++, dostosowana do systemu Windows. Z tego powodu zawiera wiele dedykowanych bibliotek dla tego systemu operacyjnego, które nie mają swoich odpowiednikach w systemach Linux, Unix lub Mac OS X. Dodatkowo Microsoft wydając ten kompilator stworzył swój własny standard języka, który do wersji Visual Studio 2013 nie był w pełni zgodny z ISO. Co za tym idzie, kod silnika powstały w wersji Visual Studio 2010 nie musi być zgodny z ISO.

\section{OpenCV}

Do ekstrakcji cech oraz do klasyfikacji wykorzystano OpenCV. Jest to biblioteka do przetwarzania obrazów napisana w języku C, oparta na otwartym kodzie i zapoczątkowana przez firmę Intel. Charakteryzuje się wieloplatformowością, gdyż istnieją wersje na systemy operacyjne Windows, Mac OS X oraz Linux. Umożliwia ponadto, dzięki zastosowaniu odpowiednich nakładek, na tworzenie kodu w różnych językach programowania: C++, C\# oraz Python. Do realizacji zadania wykorzystano OpenCV w wersji 2.2.0.

\section{Architektura systemu}
...

\section{Konfiguracja środowiska}
%TODO tutaj opiszę na czym polega konf. srodowiska, screenshoty, path itd. instalacja opencv
...


\section{Opis API}
...