\chapter{Implementacja}

Silnik kategoryzacyjny został zaimplementowany w języku C++, z użyciem środowiska Visual Studio w wersji 2010 Express, która jest czasowo ograniczona i bezpłatna do zastosowań niekomercyjnych.

\section{Język C++ i Visual Studio}

Język C++ jest językiem programowania ogólnego przeznaczenia. Jest językiem wieloparadygmatowym, czyli umożliwia stosowanie różnych paradygmatów programowania: proceduralnego, obiektowego i generycznego. Został zaprojektowany przez Bjarne Stroustrupa jako rozszerzenie języka C i użyty po raz pierwszy w 1979 roku. Charakteryzuje się obiektowymi mechanizmami abstrakcji danych oraz silną statyczną kontrolą typów.

Początkowo do realizacji zadania miał zostać wykorzystany język Python, ze względu na bogactwo bibliotek do uczenia maszynowego oraz przetwarzania obrazów. Warto wspomnieć szczególnie o dwóch bibliotekach: scikit-learn oraz scikit-image, które pozwalają na rozwiązanie wielu problemów związanych z klasyfikacją obrazów, klastrowaniem lub ekstrakcją cech. Zdecydowano o użyciu języka C++ z dwóch powodów:

\begin{compactitem}
	\item \emph{szybkość działania} -- kod jest kompilowany do kodu maszynowego, natomiast Python jest językiem interpretowanym i przez to czas potrzebny na osiągnięcie podobnych wyników jest większy
	\item \emph{silne typowanie} -- w przeciwieństwie do Pythona, język C++ jest silnie typowany, co w ocenie autora niniejszej pracy ma wpływ na zachowanie porządku w kodzie źródłowym
\end{compactitem}

Microsoft Visual Studio jest zintegrowanym środowiskiem programistycznym rozwijanym przez firmę Microsoft. 

\section{OpenCV}




\section{Architektura systemu}
...

\section{Opis API}
...