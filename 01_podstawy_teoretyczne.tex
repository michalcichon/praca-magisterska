\chapter{Podstawy teoretyczne}

Problem kategoryzacji obrazów wpisuje się w dziedzinę rozpoznawania obrazów. Zadanie to polega na rozpoznaniu przynależności różnych rodzajów obiektów do pewnych klas\cite{Tad91} i jest częścią większego zagadnienia, które określa się jako uczenie maszynowe.

Uczenie maszynowe 

%Uczenie maszynowe jest zagadnieniem bardzo szerokim i zasadniczo można je podzielić na dwie kategorie: uczenie nadzorowane i nienadzorowane.
%TODO co to jest uczenie maszynowe?
%Uczenie maszynowe jest procesem, który polega na 

%Rozpoznawanie to jest dokonywane w przypadku braku apriorycznej informacji na temat reguł przynależności obiektów do poszczególnych klas.

\section{Uczenie nadzorowane i nienadzorowane}


\section{Ekstrakcja cech}
%TODO problem ograniczania wielkosci wektora cech
%TODO podac przyklady z literatury jak robiono ekstrakcje poprzednio

\section{Klasyfikacja}
%TODO wymienic rozne rodzaje klasyfikatorow
%https://www.youtube.com/watch?v=qdDHp29QVdw

	\subsection{Klasyfikator według funkcji potencjału}
	
	\subsection{Klasyfikator statystyczny Bayesa}
	
	\subsection{Klasyfikator według minimalnej odległości}
	
	\subsection{Klasyfikator k Najbliższych Sąsiadów}

	\subsection{Maszyna wektorów wspierających (SVM)}
	
\section{Ocena klasyfikatorów}

\section{Porównanie klasyfikatorów}