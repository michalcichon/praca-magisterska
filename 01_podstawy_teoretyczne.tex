\chapter{Podstawy teoretyczne}

Problem kategoryzacji obrazów wpisuje się w dziedzinę rozpoznawania obrazów. Zadanie to polega na rozpoznaniu przynależności różnych rodzajów obiektów do pewnych klas\cite{Tad91} i jest częścią większego zagadnienia, które określa się jako uczenie maszynowe.

\section{Uczenie maszynowe}

Uczenie maszynowe jest zagadnieniem interdyscyplinarnym z pogranicza informatyki, statystyki i sztucznej inteligencji, które zajmuje się tworzeniem systemów mogących doskonalić się za pomocą dostarczanych danych.



% Q: O jakie uczenie chodzi?




% Q: Co to jest training set?

% Q: Co to jest model przewidujacy?

% Q: 

\section{Uczenie nadzorowane i nienadzorowane}


\section{Rozpoznawanie obrazów}
%TODO ????

\section{Ekstrakcja cech}
%TODO problem ograniczania wielkosci wektora cech
%TODO podac przyklady z literatury jak robiono ekstrakcje poprzednio

\section{Klasyfikacja}
%TODO wymienic rozne rodzaje klasyfikatorow
%https://www.youtube.com/watch?v=qdDHp29QVdw

	\subsection{Klasyfikator według funkcji potencjału}
	
	\subsection{Klasyfikator statystyczny Bayesa}
	
	\subsection{Klasyfikator według minimalnej odległości}
	
	\subsection{Klasyfikator k Najbliższych Sąsiadów}

	\subsection{Maszyna wektorów wspierających (SVM)}
	
\section{Ocena klasyfikatorów}

%\section{Porównanie klasyfikatorów}