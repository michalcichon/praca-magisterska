\chapter{Podstawy teoretyczne}

Problem kategoryzacji obrazów wpisuje się w dziedzinę rozpoznawania obrazów. Zadanie to polega na rozpoznaniu przynależności różnych rodzajów obiektów do pewnych klas i jest częścią większego zagadnienia, które określamy jako uczenie maszynowe.

%Rozpoznawanie to jest dokonywane w przypadku braku apriorycznej informacji na temat reguł przynależności obiektów do poszczególnych klas.

Uczenie maszynowe jest zagadnieniem bardzo szerokim i zasadniczo można je podzielić na dwie kategorie: uczenie nadzorowane i nienadzorowane. 

%TODO z racji tego, że robimy taka prace jaka robimy koncentrujemy sie na uczeniu nadzorowanym, poniewaz wiemy co chcemy odnalezc - nasze predefiniowane kategorie. Nie szukamy niczego "nowego"


\section{Ekstrakcja cech}
%TODO Napisać o ekstrakcji cech

\section{Klasyfikacja}
