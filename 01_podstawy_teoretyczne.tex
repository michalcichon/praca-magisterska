\chapter{Podstawy teoretyczne}

Problem kategoryzacji obrazów wpisuje się w dziedzinę rozpoznawania obrazów. Zadanie to polega na rozpoznaniu przynależności różnych rodzajów obiektów do pewnych klas\cite{Tad91} i jest częścią większego zagadnienia, które określa się jako uczenie maszynowe.

\section{Uczenie maszynowe}

Uczenie maszynowe jest zagadnieniem interdyscyplinarnym z pogranicza informatyki i sztucznej inteligencji, które zajmuje się tworzeniem systemów mogących doskonalić się za pomocą dostarczanych danych. 

W odróżnieniu do klasycznego podejścia, gdzie komputer dostarczał wyniku poprzez wykonywanie algorytmu instrukcja po instrukcji, systemy uczące się nabywają nowej wiedzy poprzez analizowanie danych wejściowych. Nowa wiedza, która powstaje poprzez tę analizę, pozwala na wykonywanie takich instrukcji na danych, które pojawią się w przyszłości, aby wynik był coraz bardziej zbliżony do pożądanego.

\section{Uczenie nadzorowane i nienadzorowane}


\section{Ekstrakcja cech}
%TODO problem ograniczania wielkosci wektora cech
%TODO podac przyklady z literatury jak robiono ekstrakcje poprzednio

\section{Klasyfikacja}
%TODO wymienic rozne rodzaje klasyfikatorow
%https://www.youtube.com/watch?v=qdDHp29QVdw

	\subsection{Klasyfikator według funkcji potencjału}
	
	\subsection{Klasyfikator statystyczny Bayesa}
	
	\subsection{Klasyfikator według minimalnej odległości}
	
	\subsection{Klasyfikator k Najbliższych Sąsiadów}

	\subsection{Maszyna wektorów wspierających (SVM)}
	
\section{Ocena klasyfikatorów}

%\section{Porównanie klasyfikatorów}